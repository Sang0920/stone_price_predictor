\documentclass[11pt,a4paper]{article}

\usepackage[a4paper,margin=1in]{geometry}

% --- FIX: Use T5 encoding for correct Vietnamese rendering ---
\usepackage[utf8]{inputenc} 
\usepackage[T5]{fontenc}    % Changed from T1 to T5 for Vietnamese support
\usepackage{lmodern}        % Ensures a font that supports T5 is used
% -------------------------------------------------------------

\usepackage{amsmath,amssymb}
\usepackage{booktabs}
\usepackage{longtable}
\usepackage{hyperref}

\title{Special Product Identification \& Volume (m$^3$) Calculation Reference}
\author{Automated analysis}
\date{22 January 2026}

\begin{document}
\maketitle

\section{Executive summary}
This report documents (1) how ``special products'' were identified and filtered from the provided datasets, and (2) the geometry logic used to calculate volume in cubic meters (m$^3$) for each special shape code.

\section{Source files and outputs}
\subsection{Inputs}
\begin{itemize}
  \item \texttt{Code Rule AND Product list -1 (new 2025).xlsx}
  \begin{itemize}
    \item Worksheet \texttt{Mã NVL, SP, Gia công}: provides the authoritative list of special-shape codes (column \texttt{Ký hiệu.2}).
    \item Worksheet \texttt{DANH MUC SP}: product master list used to detect special products by product-code suffix.
  \end{itemize}
  \item \texttt{output.csv}: transactional/line-level dataset containing \texttt{Product\_\_r.StockKeepingUnit} and quantities.
\end{itemize}

\subsection{Generated outputs}
\begin{itemize}
  \item \texttt{special\_products\_filtered.csv} and \texttt{special\_products\_filtered.xlsx}: all rows from \texttt{output.csv} classified as special products, with added columns \texttt{special\_code} and \texttt{special\_shape}.
  \item \texttt{special\_products\_DANH\_MUC\_SP.xlsx}: all rows from worksheet \texttt{DANH MUC SP} classified as special products, with added columns \texttt{special\_code} and \texttt{special\_shape}.
\end{itemize}

\subsection{Dataset sizes}
\begin{center}
\begin{tabular}{lrr}
\toprule
Dataset & Rows & Notes \\
\midrule
\texttt{output.csv} & 3301 & 1100 unique SKUs \\
Filtered special rows (from \texttt{output.csv}) & 100 & 60 unique SKUs \\
\texttt{DANH MUC SP} worksheet & 526 & product master list \\
Filtered special rows (from \texttt{DANH MUC SP}) & 31 & special products in master list \\
\bottomrule
\end{tabular}
\end{center}

In \texttt{output.csv}, special products represent 3.0\% of rows and 5.5\% of unique SKUs.
In \texttt{DANH MUC SP}, special products represent 5.9\% of rows.

\section{Special-product definition (authoritative code list)}
The worksheet \texttt{Mã NVL, SP, Gia công} defines ``Sản phẩm đặc biệt'' (special products) via the following codes:

\begin{center}
\begin{tabular}{cll}
\toprule
Code & Vietnamese name & English interpretation \\
\midrule
L & Cắt chữ L & L-shape / L-profile \\
U & Cắt chữ U & U-shape / channel profile \\
G & Cắt góc độ & Angle cut / chamfer (degree cut) \\
C & Cắt vòng cung & Arc cut / rounded corner \\
K & Lỗ khoan & Drilled hole \\
T & Hình trụ & Cylinder \\
B & Đá bộ & Set / kit (multiple pieces) \\
V & Đá vành & Ring / annulus / ring-step stone \\
\bottomrule
\end{tabular}
\end{center}

\section{Filtering methodology}
\subsection{Filtering special products in \texttt{output.csv}}
\paragraph{Primary key.} The detection key is the SKU field \texttt{Product\_\_r.StockKeepingUnit}.
\paragraph{Observation.} SKUs in \texttt{output.csv} follow a structured pattern that typically includes a dash (``\texttt{-}'') separating a prefix (stone/finish family) from a dimension and feature segment.
Special-shape markers appear in this segment.

\paragraph{Rule.}
\begin{enumerate}
  \item Split SKU on the first dash and take the substring to the right (the ``dimension segment'').
  \item Assign a \texttt{special\_code} if the dimension segment contains a recognized special marker.
  \item \textbf{Important precedence to avoid false positives:} for U-profiles, the dimension segment commonly contains \texttt{UL}, \texttt{UR}, or \texttt{UT} (direction/variant markers).
  These include the letters \texttt{L} and/or \texttt{T}, but should be classified as \texttt{U}.
  Therefore, U-profile detection is evaluated before standalone \texttt{L} or \texttt{T} detection.
  \item If a \texttt{special\_code} is assigned, the row is kept as a special product.
\end{enumerate}

\paragraph{Implementation note.} The resulting exports include two added columns:
\begin{itemize}
  \item \texttt{special\_code}: one of \{L,U,G,C,K,T,B,V\}
  \item \texttt{special\_shape}: a human-readable label (Vietnamese + brief English hint)
\end{itemize}

\subsection{Filtering special products in \texttt{DANH MUC SP}}
For the master list worksheet \texttt{DANH MUC SP}, special products are indicated by a suffix character in \texttt{Mã sản phẩm}.
The detection used:
\begin{enumerate}
  \item Take the last character of \texttt{Mã sản phẩm}.
  \item If it is in \{L,U,G,C,K,T,B,V\}, classify the row as a special product and assign \texttt{special\_code}.
\end{enumerate}

\section{Results}
\subsection{Results for \texttt{output.csv}}
A total of 100 rows were classified as special products, corresponding to 60 unique SKUs.
\begin{center}
\begin{tabular}{cllrrr}
\toprule
Code & Vietnamese name & English interpretation & Rows & \% of special & Unique SKUs \\
\midrule
L & Cắt chữ L & L-shape / L-profile & 11 & 11.0\% & 8 \\
U & Cắt chữ U & U-shape / channel profile & 22 & 22.0\% & 18 \\
G & Cắt góc độ & Angle cut / chamfer (degree cut) & 2 & 2.0\% & 2 \\
C & Cắt vòng cung & Arc cut / rounded corner & 6 & 6.0\% & 3 \\
K & Lỗ khoan & Drilled hole & 0 & 0.0\% & 0 \\
T & Hình trụ & Cylinder & 6 & 6.0\% & 5 \\
B & Đá bộ & Set / kit (multiple pieces) & 36 & 36.0\% & 14 \\
V & Đá vành & Ring / annulus / ring-step stone & 17 & 17.0\% & 10 \\
\bottomrule
\end{tabular}
\end{center}

\noindent \textbf{Key finding:} Code \texttt{K} (Lỗ khoan) did not appear in the provided \texttt{output.csv}.
\subsection{Results for \texttt{DANH MUC SP}}
In the worksheet \texttt{DANH MUC SP}, %(danh_special_rows)d rows were classified as special products (all with code \texttt{V} in this dataset).
\section{Volume (m$^3$) calculation reference}
\subsection{Units and conventions}
All formulas below assume inputs are in meters.
If source dimensions are in millimeters (mm) or centimeters (cm), convert first:
\[
L_{m}=\frac{L_{mm}}{1000}\quad\text{or}\quad L_{m}=\frac{L_{cm}}{100},
\]
and similarly for width ($W$), height/thickness ($H$), radii ($r$), and diameters ($d$).
\subsection{Baseline: rectangular prism}
For a standard rectangular block:
\[
V = L\,W\,H.
\]

\subsection{L --- Cắt chữ L (L-shape)}
Two common interpretations are used in practice:

\paragraph{(A) L-profile cross-section (``wrap'') extruded along length.}
Let $L$ be the run length, $a$ the tread (horizontal leg), $b$ the riser (vertical leg), and $t$ the wall thickness.
The cross-section area is:
\[
A = t a + t b - t^2,
\]
and the volume is $V = L\,A$.
\paragraph{(B) L-shape in plan view (two rectangles joined) with uniform thickness.}
If plan area is the union of two rectangles minus overlap:
\[
A = (L_1W_1) + (L_2W_2) - (L_oW_o),
\]
then $V = A\,H$.
\subsection{U --- Cắt chữ U (U-profile / channel)}
Let $L$ be length, outer width $W$, outer height $H$, and wall thickness $t$.
Inner dimensions are:
\[
W_{in}=W-2t,\qquad H_{in}=H-t.
\]
Cross-section area:
\[
A = WH - W_{in}H_{in},
\]
and volume $V = L\,A$.
\subsection{G --- Cắt góc độ (angle cut)}
If one corner of a rectangle is cut away by a right-triangle of legs $a$ and $b$ (in plan view), removed area is:
\[
A_{cut} = \frac{1}{2}ab.
\]
Then:
\[
V = (LW - A_{cut})\,H.
\]

\subsection{C --- Cắt vòng cung (arc / rounded corner)}
If a quarter-circle of radius $r$ is removed from a corner (plan view), removed area is:
\[
A_{cut} = \frac{\pi r^2}{4{4}}.
\]
Then:
\[
V = (LW - A_{cut})\,H.
\]
More generally, for a sector of angle $\theta$ degrees: $A_{cut}=\pi r^2\left(\frac{\theta}{360}\right)$.
\subsection{K --- Lỗ khoan (drilled hole)}
Compute the base volume and subtract cylindrical holes.
For $n$ holes of diameter $d$ and hole depth $h$:
\[
V = LWH - n\left(\pi\left(\frac{d}{2}\right)^2 h\right).
\]
If the hole is through-thickness, use $h=H$.

\subsection{T --- Hình trụ (cylinder)}
\paragraph{Solid cylinder.}
For diameter $d$ and height $h$:
\[
V = \pi\left(\frac{d}{2}\right)^2 h.
\]
\paragraph{Hollow cylinder (pipe).}
With outer radius $R_o$ and inner radius $R_i$:
\[
V = \pi(R_o^2 - R_i^2)h.
\]

\subsection{B --- Đá bộ (set / kit)}
A set is a sum of component volumes.
If a set contains pieces $i=1\ldots k$ with per-piece volume $V_i$ and quantity $q_i$:
\[
V_{\text{set}} = \sum_{i=1}^{k} V_i q_i.
\]
Total volume for $N$ sets: $V_{\text{total}} = N\,V_{\text{set}}$.

\subsection{V --- Đá vành (ring / annulus)}
For a full ring (annulus) with outer radius $R_o$, inner radius $R_i$, and thickness $H$:
\[
V = \pi(R_o^2 - R_i^2)H.
\]
For a partial ring (arc) spanning $\theta$ degrees:
\[
V = \pi(R_o^2 - R_i^2)H\left(\frac{\theta}{360}\right).
\]

\section{Known data gaps and recommendations}
\begin{itemize}
  \item Several U-profile lines in \texttt{output.csv} have missing height/thickness fields.
  A reliable m$^3$ calculation for U-profiles requires the missing profile depth and wall thickness (or a drawing/specification reference).
  \item Some SKUs contain multiple letters that could be interpreted as special markers (e.g., \texttt{UL} or \texttt{UT});
  ensure extraction logic uses context/precedence so that direction markers do not override the intended special code.
  \item For \texttt{B} (set) products, m$^3$ should be calculated from the bill of materials (piece list) rather than any single dimension string.
\end{itemize}

\appendix
\section{Appendix: practical extraction pseudocode}
\begin{verbatim}
For each row in output.csv:
  sku = Product__r.StockKeepingUnit
  dim = substring_after_first_dash(sku)

  if dim contains "U" immediately followed by L/R/T (e.g., UL, UR, UT): code="U"
  else if dim contains "B": code="B"
  else if dim contains "V": code="V"
  else if dim contains "L": code="L"
  else if dim contains "C": code="C"
  else if dim contains "G": code="G"
  else if dim contains "K": code="K"
  else if dim contains "T" (or "(T)"): code="T"
  else: code=None

  keep row if code is not None
\end{verbatim}

\end{document}