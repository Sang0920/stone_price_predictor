\documentclass[a4paper,12pt]{article}
\usepackage[utf8]{inputenc}
\usepackage[T5]{fontenc} % Essential for Vietnamese font rendering
\usepackage[vietnamese]{babel}
\usepackage{geometry}
\usepackage{booktabs} % For nicer tables
\usepackage{longtable} % For tables that span pages
\usepackage{array}
\usepackage{amsmath}

\geometry{left=2cm, right=2cm, top=2cm, bottom=2cm}

\title{\textbf{QUY ĐỊNH MÃ SKU SẢN PHẨM}}
\author{}
\date{}

\begin{document}

\maketitle

\section{Nguyên tắc đặt mã sản phẩm}
% Source: [cite: 1, 2, 4, 5]

Mã sản phẩm bao gồm chuỗi các ký tự và số được sắp xếp theo trật tự logic để quản lý vận hành. Cấu trúc mã được quy định như sau:

\begin{longtable}{|p{1.5cm}|p{3.5cm}|p{9cm}|}
\hline
\textbf{Vị trí} & \textbf{Định dạng} & \textbf{Mô tả} \\
\hline
\endhead

1, 2 & 2 chữ cái & \textbf{Nguyên vật liệu} (Mã loại đá). \\
\hline
3, 4 & 2 số & \textbf{Mục đích sử dụng}. \\
\hline
5, 6, 7 & 3 chữ cái & \textbf{Mã gia công bề mặt chính}. \\
\hline
8 & 1 số & \textbf{Mã phân biệt gia công phụ}. \\
\hline
9, 10, 11, 12 & 4 số (mm) & \textbf{Chiều dài}. \\
\hline
13 & 1 số hoặc 1 chữ & \textbf{Chiều rộng}: 1 số đối với đá hình hộp hoặc 1 chữ cái đối với sản phẩm đặc biệt (Vành, trụ, chữ L, chữ U,...). \\
\hline
14, 15, 16 & 3 số (mm) & \textbf{Chiều cao}. \\
\hline
\end{longtable}

\subsection{Ví dụ minh họa}
% Source: 

\textbf{Mã SKU: BD01DOT2-06004060}

\begin{itemize}
    \item \textbf{BD}: Đá bazan đen (Nguyên vật liệu).
    \item \textbf{01}: Đá lát (Mục đích sử dụng).
    \item \textbf{DOT}: Mặt đốt (Gia công bề mặt chính).
    \item \textbf{2}: Cạnh chẻ tay tự nhiên (Gia công phụ).
    \item \textbf{-}: Dấu gạch nối phân cách.
    \item \textbf{0600}: Chiều dài 600mm.
    \item \textbf{4}: Ký hiệu chiều rộng 400mm.
    \item \textbf{060}: Chiều cao 60mm.
\end{itemize}

\textbf{Diễn giải:} BAZAN ĐEN LÁT NỀN, MẶT ĐỐT, CẠNH CHẺ TAY TỰ NHIÊN, KT 600x400x60 mm.

\section{Cách gọi tên tiếng Việt}
% Source: 

Nguyên tắc diễn giải tên sản phẩm theo thứ tự logic:

\begin{enumerate}
    \item \textbf{Sản phẩm phổ thông:} Nguyên liệu + Tên sản phẩm + Gia công mặt + Cạnh/Đáy.
    \item \textbf{Sản phẩm cao cấp/phức tạp:} Nguyên liệu + Tên sản phẩm + Gia công mặt + Cạnh + Đáy + Mép + Gia công khác + Hình vẽ (nếu có).
\end{enumerate}

\textbf{Ví dụ:}
\begin{itemize}
    \item \textit{Phổ thông:} Bazan đen Lát nền, mặt Đốt, cạnh Chẻ tay, đáy Cưa[cite: 14].
    \item \textit{Cao cấp:} Bazan xám Hồ bơi, mặt Đốt, 1 cạnh dài Hon, các cạnh còn lại Cưa, đáy Xẻ rãnh 0.7x0.7cm theo chiều dài, các mép trên Vạt 3mm, Khoan 2 đầu, theo Bản vẽ đính kèm[cite: 16].
\end{itemize}

\section{Bảng ký hiệu tra cứu}

\subsection{Nguyên vật liệu (Vị trí 1, 2)}
% Source: 
\begin{longtable}{llp{5cm}}
\toprule
\textbf{Ký hiệu} & \textbf{Tên tiếng Việt} & \textbf{Tên tiếng Anh} \\
\midrule
MB & Marble Bluestone & Marble Blue \\
MT & Marble Trắng & Marble White \\
MV & Marble Vàng & Marble Yellow \\
BT & Đá Bazan tổ ong & Basalt Hive \\
BX & Đá Bazan Xám & Basalt Grey \\
BD & Đá Bazan Đen & Basalt Black \\
GX & Đá Granite Xám & Granite Grey \\
GT & Đá Granite Trắng & Granite White \\
GV & Đá Granite Vàng & Granite Yellow \\
GD & Đá Granite Đỏ & Granite Red \\
GH & Đá Granite Hồng & Granite Pink \\
\bottomrule
\end{longtable}

\subsection{Mục đích sử dụng (Vị trí 3, 4)}
% Source: 
\begin{longtable}{lp{10cm}}
\toprule
\textbf{Ký hiệu} & \textbf{Mô tả} \\
\midrule
01 & Đá lát nền ngoại thất (Cubic, tấm) \\
02 & Tường rào (Đá khối, NTR) \\
03 & Đá cây \\
04 & Đá bậc thang (Nguyên khối, ốp BT) \\
05 & Đá mỹ nghệ \\
06 & Cao cấp hồ bơi, bộ cửa \\
07 & Lát nền bên trong, đá bộ \\
08 & Ốp tường \\
09 & Slab, bàn bếp, cao cấp \\
\bottomrule
\end{longtable}

\subsection{Gia công bề mặt chính (Vị trí 5, 6, 7)}
% Source: 
\begin{longtable}{ll}
\toprule
\textbf{Ký hiệu} & \textbf{Gia công bề mặt} \\
\midrule
CTA & Chẻ tay tự nhiên \\
CUA & Mặt cưa \\
CLO & Cưa lột tay \\
TDE & Tẩy đẹp \\
DOT & Mặt đốt \\
HON & Mặt hon \\
BON & Mặt bóng \\
BAM & Mặt băm \\
GCR & Giả cổ rung \\
GCT & Giả cổ tay \\
MGI & Mài giấy \\
PCA & Mặt phun cát \\
MCA & Mài cát \\
QME & Quay mẻ \\
TLO & Tách lồi \\
\bottomrule
\end{longtable}

\subsection{Gia công phụ (Vị trí 8)}
% Source: 
\begin{longtable}{lp{8cm}}
\toprule
\textbf{Ký hiệu} & \textbf{Gia công phụ} \\
\midrule
0 & Không có gia công phụ \\
1 & Cạnh cưa \\
2 & Cạnh chẻ tay tự nhiên \\
3 & Cạnh hone \\
4 & Cạnh đốt \\
5 & Cạnh băm \\
6 & Cạnh bo tròn R \\
7 & Đáy băm \\
8 & Gõ mẻ \\
9 & Gia công khác (Có chú thích) \\
\bottomrule
\end{longtable}

\subsection{Sản phẩm đặc biệt (Vị trí 13)}
% Source: 
\textit{Lưu ý: Sử dụng khi sản phẩm không phải hình khối chữ nhật tiêu chuẩn hoặc có yêu cầu đặc biệt.}

\begin{longtable}{ll}
\toprule
\textbf{Ký hiệu} & \textbf{Sản phẩm đặc biệt} \\
\midrule
R & Xẻ rãnh thoát nước \\
L & Cắt chữ L \\
U & Cắt chữ U \\
G & Cắt góc vuông \\
C & Cắt vòng cung \\
K & Lỗ khoan \\
T & Hình trụ \\
B & Đá bộ \\
V & Đá vành \\
\bottomrule
\end{longtable}

\end{document}